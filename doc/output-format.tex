\documentclass{article}

\usepackage{amsmath}
\usepackage{amssymb}
\usepackage{natbib}
\usepackage{longtable}
\usepackage{cleveref}
\usepackage{lscape}

\crefformat{footnote}{#2\footnotemark[#1]#3}
\crefmultiformat{footnote}{#2\footnotemark[#1]#3}
                          {$^,$#2\footnotemark[#1]#3}
                          {$^,$#2\footnotemark[#1]#3}
                          {$^,$#2\footnotemark[#1]#3}

% Page size

\special{papersize=8.5in,11in}
\setlength{\pdfpageheight}{\paperheight}
\setlength{\pdfpagewidth}{\paperwidth}

\setlength{\textwidth}{6.5in} 
\setlength{\textheight}{9in}
\setlength{\topmargin}{0in} 
\setlength{\oddsidemargin}{0in}
\setlength{\evensidemargin}{0in} 
\setlength{\headheight}{0in}
\setlength{\headsep}{0in} 
\setlength{\hoffset}{0in}
\setlength{\voffset}{0in}

\setlength{\abovecaptionskip}{0pt}
\setlength{\belowcaptionskip}{0pt} 

\setlength{\tabcolsep}{1em}

% Macros

\newcommand{\diff}{\ensuremath{{\rm d}}}

\newcommand{\Rstar}{\ensuremath{R_{\ast}}}
\newcommand{\Mstar}{\ensuremath{M_{\ast}}}
\newcommand{\Lrad}{\ensuremath{L_{\rm rad}}}
\newcommand{\Lstar}{\ensuremath{L_{\ast}}}

\newcommand{\cm}{\ensuremath{{\rm cm}}}
\newcommand{\gram}{\ensuremath{{\rm g}}}
\newcommand{\second}{\ensuremath{{\rm s}}}
\newcommand{\dyne}{\ensuremath{{\rm dyn}}}
\newcommand{\erg}{\ensuremath{{\rm erg}}}
\newcommand{\kelvin}{\ensuremath{{\rm K}}}

\begin{document}

\section*{Output File Formats}

\subsection*{Summary Files}

Summary data for all modes found by GYRE are stored in an HDF5-format
file. The \texttt{summary\_item\_names} parameter controls which variables
are written to the file; it is a comma-separated list of item names
drawn from the table below. Scalar items are stored as attributes,
while 1-D array items are stored as datasets.

\begin{center}
\begin{longtable}{cllll} \hline
Variable & Units & Item Name & Item type\footnote{%
\label{foot:data-type}
Real attributes and datasets are written with type \texttt{H5T\_IEEE\_F64LE}. 
Integer attributes and datasets are written with type \texttt{H5T\_STD\_I64LE}. 
Complex attributes and datasets are written as a compound type, composed of a real component \texttt{re} and an imaginary component \texttt{im}, both with type \texttt{H5T\_IEEE\_F64LE}.} & 
Definition \\ \hline
$\ell$ & --- & \texttt{l} & integer dataset & Harmonic degrees \\
$n_{\rm p}$ & --- & \texttt{n\_p} & integer dataset & p-mode radial orders \\
$n_{\rm g}$ & --- & \texttt{n\_g} & integer dataset & g-mode radial orders \\
$\omega$ & ---   & \texttt{omega} & complex dataset & Dimensionless angular eigenfrequencies \\
$f$      & \emph{varies}\footnote{%
\label{foot:freq-units}
The units of $f$ depend on the value of the
\texttt{freq\_units} field in the \texttt{\&output} namelist.} & 
\texttt{freq} & complex dataset & Generic eigenfrequencies \\
$E$ & --- & \texttt{E} & real dataset & Normalized mode inertias\footnote{\label{foot:inertia}See \citet[][his eqn.~13]{ChrDal2012}.} \\
$K$ & $G\Mstar^{2}/\Rstar$ & \texttt{K} & real dataset & Total kinetic energies \\
$W$\footnote{%
\label{foot:coeffs-nad}
Only available from \texttt{gyre\_nad}} & $\Lstar \sqrt{\Rstar^{3}/G\Mstar}$ & \texttt{W} & real dataset & Total works \\
\Mstar\footnote{%
\label{foot:coeffs-evol}
Available when \texttt{coeffs\_type} is \texttt{EVOL}} & 
\gram & \texttt{M\_star} & real attribute & Stellar mass \\
\Rstar\cref{foot:coeffs-evol} & \cm & \texttt{R\_star} & real attribute & Stellar radius \\
\Lstar\cref{foot:coeffs-evol} & $\erg\,\second^{-1}$ & \texttt{L\_star} & real attribute & Stellar luminosity \\
$n_{\rm poly}$\footnote{%
\label{foot:coeffs-poly}
Only available when \texttt{coeffs\_type} is \texttt{POLY}.} & 
--- & \texttt{n\_poly} & real attribute & Polytropic index \\ \hline
\caption{Output data for summary files}
\end{longtable}
\end{center}

\newpage

\subsection*{Mode Files}

Detailed data for each individual mode found by GYRE are stored in HDF5-format
files. The \texttt{mode\_item\_names} parameter controls which variables
are written to the files; it is a comma-separated list of item names
drawn from the table below. Scalar items are stored as attributes,
while 1-D array items are stored as datasets.

\begin{landscape}
\begin{center}
\begin{longtable}{cllll} \hline
Variable & Units & Item Name & Item type\cref{foot:data-type} & Definition \\ \hline
$n$ & --- & \texttt{n} & integer attribute & Number of grid points \\
$\ell$ & --- & \texttt{l} & integer attribute & Harmonic degree \\
$n_{\rm p}$ & --- & \texttt{n\_p} & integer attribute & p-mode radial order \\
$n_{\rm g}$ & --- & \texttt{n\_g} & integer attribute & g-mode radial order \\
$\omega$ & ---   & \texttt{omega} & complex attribute & Dimensionless angular eigenfrequency \\
$f$      & \emph{varies}\cref{foot:freq-units} & \texttt{freq} & complex attribute & Generic eigenfrequency \\
$E$ & --- & \texttt{E} & real attribute & Normalized mode inertia\cref{foot:inertia} \\
$K$ & $G\Mstar^{2}/\Rstar$ & \texttt{K} & real attribute & Kinetic energy \\ 
$W$ & $G\Mstar^{2}/\Rstar$ & \texttt{dW\_dx} & real attribute & Work \\
$x$ & --- & \texttt{x} & real dataset & $r/\Rstar$ \\
$V$ & --- & \texttt{V} & real dataset & $-\diff \ln p/\diff \ln r$ \\
$A^{\ast}$ & --- & \texttt{As} & real dataset & $\Gamma_{1}^{-1} \diff \ln p/\diff \ln r - \diff \ln \rho/\diff \ln r$ \\
$U$ & --- & \texttt{U} & real dataset & $\diff \ln M_{r}/\diff \ln r$ \\
$c_{1}$ & --- & \texttt{c\_1} & real dataset & $(r/\Rstar)^{3} (\Mstar/M_{r})$ \\
$\Gamma_{1}$ & --- & \texttt{Gamma\_1} & real dataset & $(\partial \ln p/\partial \ln \rho)_{\rm ad}$ \\
$\xi_{r}$ & \Rstar & \texttt{xi\_r} & complex dataset & Radial displacement perturbation \\
$\xi_{h}$ & \Rstar & \texttt{xi\_h} & complex dataset & Horizontal displacement perturbation \\
$\phi'$ & $G\Mstar/\Rstar$ & \texttt{phip} & complex dataset & Eulerian potential perturbation \\
$\diff\phi'/\diff x$ & $G\Mstar/\Rstar$ & \texttt{dphip\_dx} & complex dataset & Eulerian radial gravity perturbation \\
$\delta S$\cref{foot:coeffs-evol} & $c_{p}$ & \texttt{delS} & complex dataset & Lagrangian specific entropy perturbation \\
$\delta L$\cref{foot:coeffs-evol,foot:coeffs-poly} & \Lstar & \texttt{delL} & complex dataset & Lagrangian luminosity perturbation \\
$\delta L_{\rm qad}$\cref{foot:coeffs-evol,foot:coeffs-poly} & \Lstar & \texttt{delL\_qad} & complex dataset & quasi-adiabatic Lagrangian luminosity perturbation \\
$\delta p$\cref{foot:coeffs-evol,foot:coeffs-poly} & $p$ & \texttt{delp} & complex dataset & Lagrangian pressure perturbation \\
$\delta \rho$\cref{foot:coeffs-evol,foot:coeffs-poly} & $\rho$ & \texttt{delrho} & complex dataset & Lagrangian density perturbation \\
$\delta T$\cref{foot:coeffs-evol,foot:coeffs-poly} & $T$ & \texttt{delT} & complex dataset & Lagrangian temperature perturbation \\
$\diff K/\diff x$ & $G\Mstar^{2}/\Rstar$ & \texttt{dK\_dx} & real dataset & Differential kinetic energy \\
$\diff W/\diff x$\cref{foot:coeffs-nad} & $\Lstar \sqrt{\Rstar^{3}/G\Mstar}$ & \texttt{dW\_dx} & real dataset & Differential work \\ 
\Mstar\cref{foot:coeffs-evol} & \gram & \texttt{M\_star} & real attribute & Stellar mass \\
\Rstar\cref{foot:coeffs-evol} & \cm & \texttt{R\_star} & real attribute & Stellar radius \\
\Lstar\cref{foot:coeffs-evol} & $\erg\,\second^{-1}$ & \texttt{L\_star} & real attribute & Stellar luminosity \\
$w$\cref{foot:coeffs-evol} & --- & \texttt{w} & real dataset & $M_{r}/(\Mstar-M_{r})$ \\
$p$\cref{foot:coeffs-evol} & $\dyne\,\cm^{-2}$ & \texttt{p} & real dataset & Total pressure \\
$\rho$\cref{foot:coeffs-evol} & $\gram\,\cm^{-3}$ & \texttt{rho} & real dataset & Density \\
$T$\cref{foot:coeffs-evol} & $\kelvin$ & \texttt{T} & real dataset & Temperature \\ \hline
\caption{Output data for mode files}
\end{longtable}
\end{center}
\end{landscape}

\bibliographystyle{gyre}
\bibliography{output-format}

\end{document}
