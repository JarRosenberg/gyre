\documentclass{article}

\usepackage{amsmath}
\usepackage{amssymb}

% Page size

\special{papersize=8.5in,11in}
\setlength{\pdfpageheight}{\paperheight}
\setlength{\pdfpagewidth}{\paperwidth}

\setlength{\textwidth}{6.5in} 
\setlength{\textheight}{9in}
\setlength{\topmargin}{0in} 
\setlength{\oddsidemargin}{0in}
\setlength{\evensidemargin}{0in} 
\setlength{\headheight}{0in}
\setlength{\headsep}{0in} 
\setlength{\hoffset}{0in}
\setlength{\voffset}{0in}

\setlength{\abovecaptionskip}{0pt}
\setlength{\belowcaptionskip}{0pt} 

% Macros

\newcommand{\diff}{\ensuremath{{\rm d}}}

\newcommand{\Rstar}{\ensuremath{R_{\ast}}}
\newcommand{\Mstar}{\ensuremath{M_{\ast}}}
\newcommand{\Lrad}{\ensuremath{L_{\rm rad}}}
\newcommand{\Lstar}{\ensuremath{L_{\ast}}}

\newcommand{\cm}{\ensuremath{{\rm cm}}}
\newcommand{\gram}{\ensuremath{{\rm g}}}
\newcommand{\second}{\ensuremath{{\rm s}}}
\newcommand{\dyne}{\ensuremath{{\rm dyn}}}
\newcommand{\erg}{\ensuremath{{\rm erg}}}
\newcommand{\kelvin}{\ensuremath{{\rm K}}}

\begin{document}

\section*{GYRE Stellar Model (GSM) Format}

GSM-format files store data describing a stellar model in an
HDF5-format file. The attributes of the root group contain global
stellar parameters, while 1-D datasets contained within the root group
specify the structure data on a grid of $n$ points extending from
center to surface. These attributes and datasets are defined as
follows:

\begin{table}[h!]
\begin{tabular}{|c|l|c|l|l|} \hline
Variable & Object name & (A)ttribute / & Object datatype & Definition \\
         &             & (D)ataset     &                 &            \\ \hline
\Rstar            & \texttt{R\_star}      & A & \texttt{H5T\_IEEE\_F64LE} & Stellar radius ($\cm$) \\
\Mstar            & \texttt{M\_star}      & A & \texttt{H5T\_IEEE\_F64LE} & Stellar mass ($\gram$) \\
\Lstar            & \texttt{L\_star}      & A & \texttt{H5T\_IEEE\_F64LE} & Stellar luminosity ($\erg\,\second^{-1}$) \\
$n$               & \texttt{n}            & A &\texttt{H5T\_STD\_I64LE}  & Number of grid points \\ 
$r$               & \texttt{r}            & D & \texttt{H5T\_IEEE\_F64LE} & Radius ($\cm$) \\
$w$               & \texttt{w}            & D & \texttt{H5T\_IEEE\_F64LE} & $M_{r}/(\Mstar-M_{r})$ \\
$p$               & \texttt{p}            & D & \texttt{H5T\_IEEE\_F64LE} & Total pressure ($\dyne\,\cm^{-2}$) \\
$T$               & \texttt{T}            & D & \texttt{H5T\_IEEE\_F64LE} & Temperature ($\kelvin$) \\
$\rho$            & \texttt{rho}          & D & \texttt{H5T\_IEEE\_F64LE} & Density ($\gram\,\cm^{-2}$) \\
$N^{2}$           & \texttt{N2}           & D & \texttt{H5T\_IEEE\_F64LE} & Brunt-V\"ais\"al\"a frequency squared ($\second^{-2}$) \\
$\Gamma_{1}$      & \texttt{Gamma\_1}      & D & \texttt{H5T\_IEEE\_F64LE} & $(\partial \ln p/\partial \ln \rho)_{\rm ad}$ \\
$\nabla_{\rm ad}$  & \texttt{nabla}        & D & \texttt{H5T\_IEEE\_F64LE} & $(\diff \ln T/\diff \ln p)_{\rm ad}$ \\
$\delta$      & \texttt{delta}            & D & \texttt{H5T\_IEEE\_F64LE} & $-(\partial \ln \rho/\partial \ln T)_{p}$  \\
$\nabla$          & \texttt{nabla}        & D & \texttt{H5T\_IEEE\_F64LE} & $\diff \ln T/\diff \ln p$ \\
$\epsilon$        & \texttt{epsilon}      & D &  \texttt{H5T\_IEEE\_F64LE} & energy generation rate ($\erg\,s^{-1}\,\gram^{-1}$) \\
$\epsilon_{T}$    & \texttt{epsilon\_T}   & D &  \texttt{H5T\_IEEE\_F64LE} & $(\partial \epsilon/\partial \ln T)_{\rho}$ ($\erg\,s^{-1}\,\gram^{-1}$) \\
$\epsilon_{\rho}$ & \texttt{epsilon\_rho} & D &  \texttt{H5T\_IEEE\_F64LE} & $(\partial \epsilon/\partial \ln \rho)_{T}$ ($\erg\,s^{-1}\,\gram^{-1}$) \\
$\kappa$          & \texttt{kappa}       & D &  \texttt{H5T\_IEEE\_F64LE} & opacity ($\cm^{2}\,\gram^{-1}$) \\
$\kappa_{T}$      & \texttt{kappa\_T}     & D &  \texttt{H5T\_IEEE\_F64LE} & $(\partial \ln \kappa/\partial \ln T)_{\rho}$ \\
$\kappa_{\rho}$   & \texttt{kappa\_rho}   & D &  \texttt{H5T\_IEEE\_F64LE} & $(\partial \ln \kappa/\partial \ln \rho)_{T}$ \\
$\Omega_{\rm rot}$ & \texttt{Omega\_rot}   & D & \texttt{H5T\_IEEE\_F64LE} & Rotation angular velocity (${\rm rad}\,\second^{-1}$) \\  \hline
\end{tabular}
\end{table}

\end{document}
