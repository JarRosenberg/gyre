\documentclass[fleqn]{article}

\usepackage{amsmath}
\usepackage{amssymb}
\usepackage{natbib}

\usepackage{textcomp,upquote}
\usepackage{regexpatch}

% Page size

\special{papersize=8.5in,11in}
\setlength{\pdfpageheight}{\paperheight}
\setlength{\pdfpagewidth}{\paperwidth}

\setlength{\textwidth}{6.5in} 
\setlength{\textheight}{9in}
\setlength{\topmargin}{0in} 
\setlength{\oddsidemargin}{0in}
\setlength{\evensidemargin}{0in} 
\setlength{\headheight}{0in}
\setlength{\headsep}{0in} 
\setlength{\hoffset}{0in}
\setlength{\voffset}{0in}

\setlength{\abovecaptionskip}{0pt}
\setlength{\belowcaptionskip}{0pt} 

% Fix quote typesetting

\makeatletter
\def\active@text@prime{\ifin@texttt\textquotesingle\else'\fi}
\def\active@math@prime{^\bgroup\prim@s}
\newif\ifin@texttt

\regexpatchcmd{\pr@m@s}{\'}{\cA\'}{}{}
\xapptocmd{\ttfamily}{\in@texttttrue}{}{}

\begingroup\lccode`\~=`\'
\lowercase{\endgroup\protected\def~}{%
  \ifmmode
    \expandafter\active@math@prime
  \else
    \expandafter\active@text@prime
  \fi}
\AtBeginDocument{\catcode`\'=\active}

\begingroup
\obeylines\obeyspaces%
\gdef\@resetactivechars{%
\def^^M{\@activechar@info{EOL}\space}%
\def {\@activechar@info{space}\space}%
}%
\endgroup

% Journals

\newcommand{\actaa}{Act. Ast.}
\newcommand{\apj}{ApJ}
\newcommand{\apjs}{ApJS}
\newcommand{\apjl}{ApJ}
\newcommand{\apss}{Ap\&SS}
\newcommand{\aap}{A\&A}
\newcommand{\aaps}{A\&AS}
\newcommand{\aapr}{A\&A~Rev.}
\newcommand{\aj}{AJ}
\newcommand{\mnras}{MNRAS}
\newcommand{\araa}{ARAA}
\newcommand{\pasp}{PASP}
\newcommand{\pasa}{PASA}
\newcommand{\spie}{SPIE}

% Abbreviations

\newcommand{\gyre}{GYRE}

% Math symbols

\newcommand{\diff}{\mathrm{d}}
\newcommand{\ii}{\mathrm{i}}

\newcommand{\omegac}{\omega_{\rm c}}

\newcommand{\elle}{\ell_{\rm e}}
\newcommand{\elli}{\ell_{\rm i}}

\newcommand{\Vg}{\frac{V}{\Gamma_{1}}}
\newcommand{\As}{A^{\ast}}

\newcommand{\nabad}{\nabla_{\rm ad}}

\newcommand{\kapad}{\kappa_{\rm ad}}
\newcommand{\kapS}{\kappa_{S}}
\newcommand{\epsad}{\epsilon_{\rm ad}}
\newcommand{\epsS}{\epsilon_{S}}

\newcommand{\crad}{c_{\rm rad}}
\newcommand{\dcrad}{\partial\crad}
\newcommand{\cepsad}{c_{\epsilon,{\rm ad}}}
\newcommand{\cepsS}{c_{\epsilon,S}}
\newcommand{\cdif}{c_{\rm dif}}
\newcommand{\cthn}{c_{\rm thn}}
\newcommand{\dcthn}{\partial\cthn}
\newcommand{\cthk}{c_{\rm thk}}

\newcommand{\agr}{\alpha_{\rm gr}}
\newcommand{\ahf}{\alpha_{\rm hf}}
\newcommand{\arh}{\alpha_{\rm rh}}
\newcommand{\frh}{f_{\rm rh}}
\newcommand{\dfrh}{\partial f_{\rm rh}}

\newcommand{\Rstar}{R_{\ast}}
\newcommand{\Mstar}{M_{\ast}}
\newcommand{\Lrad}{L_{\rm rad}}
\newcommand{\Lstar}{L_{\ast}}

% Main document

\begin{document}

\begin{center}
  {\LARGE \textbf{\gyre\ Equations \& Variables}}
\end{center}

\section*{Preliminaries}

\gyre\ radial eigenfunctions are expressed in terms of a set of
dimensionless variables $y_{i}(x)$ ($i=1,2,\ldots$), where $x\equiv
r/R$ is the dimensionless radial coordinate. The equations governing
these eigenfunctions depend on the underlying stellar structure, and
on the dimensionless oscillation frequency in the co-rotating frame,
\begin{equation*}
  \omegac = \omega - m \Omega(x).
\end{equation*}
Here, $\Omega(x)$ is the rotation angular frequency, $m$ is the
azimuthal order, and $\omega$ is the corresponding dimensionless
frequency in an inertial frame.

The equations also depend on the effective harmonic degree $\elle$. In
the non-rotating limit, $\elle$ reduces to the ordinary spherical
harmonic degree $\ell$. Within the traditional approximation of
rotation (TAR), $\elle$ is obtained by solving
\begin{equation*}
  \elle(\elle+1) = \lambda(\ell,m;\nu),
\end{equation*}
where $\lambda(\ell,m;\nu)$ is the eigenvalue of Laplace's tidal
equation \citep[see, e.g.,][]{Townsend:2003aa} for the indicated
$\ell$ and $m$, and for spin parameter $\nu \equiv 2
\Omega/\omegac$. Due to its dependence on $\Omega$ (both directly, and
through $\omegac$), $\elle$ varies with position in a differentially
rotating star. The value of $\elle$ at the inner boundary is denoted
$\elli$, and the dependent variables $y_{i}$ are scaled using $\elli$
so that they approach constant values at this boundary.

\section*{Structure Coefficients}

The properties of the underlying stellar structure are described by a
set of dimensionless structure coefficients, which largely follow the
definitions in \citet{Unno:1989aa}.

\subsection*{Mechanical}

\begin{gather*}
V = -\frac{\diff \ln P}{\diff \ln r} \qquad
\As = \frac{1}{\Gamma_{1}} \frac{\diff \ln P}{\diff \ln r} - \frac{\diff \ln \rho}{\diff \ln r} \qquad
U = \frac{\diff \ln M_{r}}{\diff \ln r} \qquad
c_1 = \frac{r^{3}}{\Rstar^{3}} \frac{\Mstar}{M_{r}} \qquad
\Gamma_{1} = \left( \frac{\partial \ln P}{\partial \ln \rho} \right)_{S}
\end{gather*}

\subsection*{Thermal}

\begin{gather*}
\nabla = \frac{\diff \ln T}{\diff \ln P} \qquad
\nabad = \left( \frac{\partial \ln T}{\partial \ln P} \right)_{S} \qquad
\delta = - \left( \frac{\partial \ln \rho}{\partial \ln T} \right)_{P} \qquad \\
\crad = x^{-3} \frac{\Lrad}{\Lstar} \qquad
\partial\crad = \frac{\diff \ln \crad}{\diff \ln r} \\
\cepsad = x^{-3} \frac{4\pi r^{3} \epsad \rho}{\Lstar} \qquad
\cepsS = x^{-3} \frac{4\pi r^{3} \epsS \rho}{\Lstar} \qquad \\
\cdif = \left( \kapad - 4 \nabad \right) V \nabla + \nabad \left(V + \frac{\diff \ln \nabad}{\diff \ln r} \right) \\
\cthn = \frac{c_{p}}{a c \kappa T^{3}} \sqrt{\frac{G\Mstar}{\Rstar^{3}}} \qquad
\partial\cthn = \frac{\diff \ln \cthn}{\diff \ln r} \qquad
\cthk = x^{-3} \frac{4\pi r^{3} c_{p} T \rho}{\Lstar} \sqrt{\frac{G\Mstar}{\Rstar^{3}}} \\
\kapad = \left( \frac{\partial \ln \kappa}{\partial \ln P} \right)_{S} \qquad
\kapS = c_{p} \left( \frac{\partial \ln \kappa}{\partial S} \right)_{P} \\
\epsad = \left( \frac{\partial \epsilon}{\partial \ln P} \right)_{S} \qquad
\epsS = c_{p} \left( \frac{\partial \epsilon}{\partial S} \right)_{P}
\end{gather*}

%%

\section*{Dimensionless Variables}

For non-radial non-adiabatic calculations, \gyre\ uses a set of
six dimensionless variables:
\begin{align*}
x     &= \frac{r}{\Rstar}, \\
y_{1} &= x^{2-\elli}\, \frac{\xi_{r}}{r}, \\
y_{2} &= x^{2-\elli}\, \frac{P'}{\rho g r}, \\
y_{3} &= x^{2-\elli}\, \frac{\Phi'}{gr}, \\
y_{4} &= x^{2-\elli}\, \frac{1}{g} \frac{\diff \Phi'}{\diff r}, \\
y_{5} &= x^{2-\elli}\, \frac{\delta S}{c_{p}}, \\
y_{6} &= x^{-1-\elli}\, \frac{\delta \Lrad}{\Lstar}.
\end{align*}
Here, $\xi_{r}$ is the radial displacement perturbation, primes
indicate Eulerian perturbations, and $\delta$ denotes the Legrangian
perturbation. As discussed previously, the $x^{\ldots}$ scaling of
the variables ensures that they approach constant values at the
inner boundary.

For non-radial adiabatic calculations, only the first four variables
are used; and for radial adiabatic calculations with
\texttt{reduce\_order=.TRUE.} , only the first two.

%%

\section*{Differential Equations}

For non-radial non-adiabatic calculations, \gyre\ solves a system of
six coupled, first-order differential equations:
\begin{align*}
x \frac{\diff y_{1}}{\diff x} &=
\left(\Vg - 1 - \elli \right) y_{1} +
\left(\frac{\lambda}{c_{1} \omega^{2}} - \Vg \right) y_{2} +
\agr \frac{\lambda}{c_{1} \omega^{2}} y_{3} +
\delta y_{5}, \\
x \frac{\diff y_{2}}{\diff x} &=
(c_{1} \omega^{2} - \As ) y_{1} +
(3 - U + \As - \elli) y_{2} -
\agr y_{4} +
\delta y_{5}, \\
x \frac{\diff y_{3}}{\diff x} &=
\agr (3 - U - \elli) y_{3} +
\agr y_{4}, \\
x \frac{\diff y_{4}}{\diff x} &=
\agr \As U y_{1} +
\agr \Vg U y_{2} +
\agr \lambda y_{3} -
\agr (U + \elli - 2) y_{4}
- \agr \delta U y_{5}, \\
x \frac{\diff y_{5}}{\diff x} &=
\frac{V}{\frh} \left[ \nabad (U - c_{1}\omega^{2}) - 4 (\nabad - \nabla) + \cdif \right] y_{1} + \mbox{} \\
&
\frac{V}{\frh} \left[ \frac{\lambda}{c_{1} \omega^{2}} (\nabad - \nabla) - \cdif \right] y_{2} +
\agr \frac{V}{\frh} \left[ \frac{\lambda}{c_{1} \omega^{2}} (\nabad - \nabla) \right] y_{3} + \agr \frac{V \nabad}{\frh} y_{4} + \mbox{} \\
& 
\left[ \frac{V \nabla}{\frh} (4 \frh - \kapS) + \dfrh + 2 - \elli \right] y_{5} -
\frac{V \nabla}{\frh \crad} y_{6}, \\
x \frac{\diff y_{6}}{\diff x} &=
\left[ \ahf \lambda\left( \frac{\nabad}{\nabla} - 1 \right) \crad - V \cepsad \right] y_{1} +
\left[ V \cepsad - \lambda \crad \left( \ahf \frac{\nabad}{\nabla} - \frac{3 + \dcrad}{c_{1}\omega^{2}} \right) \right] y_{2} + \mbox{} \\
&
\agr \left[ \lambda \crad \frac{3 + \dcrad}{c_{1}\omega^{2}} \right] y_{3} +
\left[ \cepsS - \ahf \frac{\lambda\crad}{\nabla V} + \ii \omega \cthk\right] y_{5} -
\left[ 1 + \elli \right] y_{6}.
\end{align*}
The $\agr$ coefficient is set to zero in the \citet{Cowling:1941aa}
approximation (\texttt{cowling\_approx=.TRUE.}), and to one
otherwise. Likewise, the $\ahf$ coefficient is set to zero in the NARF
approximation (\texttt{narf\_approx=.TRUE.}; see
\citealp{Townsend:2005ab}), and to one otherwise. Finally,
\begin{equation}
  \frh \equiv 1 - \arh \frac{\ii \omega \cthn}{4}, \qquad \dfrh \equiv \frac{\partial \ln \frh}{\partial \ln x} = - \arh \frac{\ii \omega \cthn \dcthn}{4 \frh},
\end{equation}
with the $\arh$ set to one in the Eddington approximation
(\texttt{eddingon\_approx=.TRUE.}) and zero otherwise.

For non-radial adiabatic calculations, the last two equations are set
aside and the $y_{5}$ terms dropped from the first four equations. For
radial adiabatic calculations with \texttt{reduce\_order=.TRUE.}, the
last four equations are set aside and the first two replaced by
\begin{align*}
x \frac{\diff y_{1}}{\diff x} &=
\left(\Vg - 1 \right) y_{1} 
- \Vg y_{2}, \\
x \frac{\diff y_{2}}{\diff x} &=
(c_{1} \omega^{2} + U - \As ) y_{1} +
(3 - U + \As) y_{2}.
\end{align*}

\section*{Boundary Conditions}

\subsection*{Inner Boundary}

When \texttt{inner\_bound='REGULAR'}, \gyre\ applies
regularity-enforcing conditions at the inner boundary:
\begin{align*}
c_{1} \omega^{2} y_{1} - \ell y_{2} - \agr y_{3} &= 0, \\
\agr \ell y_{3} - (2\agr - 1) y_{4} &= 0, \\
y_{5} &= 0.
\end{align*}
When \texttt{inner\_bound='ZERO\_R'}, the first and second conditions are replaced with zero radial displacement conditions,
\begin{align*}
y_{1} &= 0, \\
y_{4} &= 0.
\end{align*}
Likewise, when \texttt{inner\_bound='ZERO\_H'}, the first and second conditions are replaced with zero horizontal displacement conditions,
\begin{align*}
y_{2} - y_{3} &= 0, \\
y_{4} &= 0.
\end{align*}


\subsection*{Outer Boundary}
 
When \texttt{outer\_bound='VACUUM'}, \gyre\ applies vacuum surface
pressure conditions at the outer boundary:
\begin{align*}
y_{1} - y_{2} &= 0 \\
\agr U y_{1} + (\agr \elle + 1) y_{3} + \agr y_{4} &= 0 \\
(2 - 4\nabad V) y_{1} + 4 \nabad V y_{2} + 4 \frh y_{5} - y_{6} &= 0
\end{align*}
When \texttt{outer\_bound='DZIEM'}, the first condition is replaced by
the \citet{Dziembowski:1971aa} outer mechanical boundary condition,
\begin{equation*}
\left\{ 1 + V^{-1} \left[ \frac{\lambda}{c_{1} \omega^{2}} - 4 - c_{1} \omega^{2} \right] \right\} y_{1} -
y_{2} = 0.
\end{equation*}
When \texttt{outer\_bound='UNNO'} or \texttt{outer\_bound='JCD'}, the
first condition is replaced by the (possibly-leaky) outer mechanical
boundary conditions described by \citet{Unno:1989aa} and
\citet{Christensen-Dalsgaard:2008ab}, respectively.
  
\section*{Jump Conditions}

Across density discontinuities, \gyre\ enforces conservation of mass,
momentum and energy by applying the jump conditions
\begin{align*}
U^{+} y_{2}^{+} - U^{-} y_{2}^{-} &= y_{1} (U^{+} - U^{-}) \\
y_{4}^{+} - y_{4}^{-} &= -y_{1} (U^{+} - U^{-}) \\
y_{5}^{+} - y_{5}^{-} &= - V^{+} \nabad^{+} (y_{2}^{+} - y_{1}) +
V^{-} \nabad^{-} (y_{2}^{-} - y_{1})
\end{align*}
Here, + (-) superscripts indicate quantities evaluated on the inner
(outer) side of the discontinuity. $y_{1}$, $y_{3}$ and $y_{6}$ remain
continuous across discontinuites, and therefore don't need these
superscripts.

\section*{Alternative Variable Sets}

\gyre\ offers the option to use different sets of dimensionless
variables, instead of the canonical set defined above. When
\texttt{variables\_set='DZIEM'}, \gyre\ uses a set based on the
formulation by \citet{Dziembowski:1971aa}:
\begin{align*}
y_{1} &= x^{2-\elli}\, \frac{\xi_{r}}{r}, \\
y_{2} &= x^{2-\elli}\, \frac{1}{g r} \left( \frac{P'}{\rho} + \Phi' \right), \\
y_{3} &= x^{2-\elli}\, \frac{\Phi'}{gr}, \\
y_{4} &= x^{2-\elli}\, \frac{1}{g} \frac{\diff \Phi'}{\diff r},
\end{align*}
with $y_{5}$ and $y_{6}$ defined as before. When
\texttt{variables\_set='JCD'}, \gyre\ uses a set based on the
formulation in the ADIPLS code \citep{Christensen-Dalsgaard:2008ab}:
\begin{align*}
y_{1} &= x^{2-\elli}\, \frac{\xi_{r}}{r}, \\
y_{2} &= x^{2-\elli}\, \frac{\lambda}{r^{2} \sigma^{2}} \left( \frac{P'}{\rho} + \Phi' \right), \\
y_{3} &= - x^{2-\elli}\, \frac{\Phi'}{gr}, \\
y_{4} &= - x^{2-\elli}\, r \frac{\diff}{\diff r} \left( \frac{\Phi'}{g r} \right),
\end{align*}
for non-radial calculations, while
\begin{align*}
y_{1} &= x^{2}\, \frac{\xi_{r}}{r}, \\
y_{2} &= x^{2}\, \frac{1}{r^{2} \sigma^{2}} \left( \frac{P'}{\rho} \right), \\
y_{3} &= - x^{2}\, \frac{\Phi'}{gr}, \\
y_{4} &= - x^{2}\, r \frac{\diff}{\diff r} \left( \frac{\Phi'}{g r} \right),
\end{align*}
in the radial case when \texttt{reduce\_order=.FALSE.}. When
\texttt{variables\_set='LAGP'}, \gyre\ uses a set which replaces the
Eulerian pressure perturbation with the Lagrangian one:
\begin{align*}
y_{1} &= x^{2-\elli}\, \frac{\xi_{r}}{r}, \\
y_{2} &= x^{ -\elli}\, \frac{\delta P}{P}, \\
y_{3} &= x^{2-\elli}\, \frac{\Phi'}{gr}, \\
y_{4} &= x^{2-\elli}\, \frac{1}{g} \frac{\diff \Phi'}{\diff r}.
\end{align*}

\bibliographystyle{gyre}
\bibliography{equations}

\end{document}
